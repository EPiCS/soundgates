\chapter{Related Work}
\label{chapter:RelatedWork}
	 
\section{Cycling '74 Max}
	Our editor is in some way similar to other music composition applications especially cycling74's MAX. Also it offers a drawing canvas, which can be filled with connected components. At any time the patch, which is the combination of every used component, can be simulated to hear the generated sound. However, this covers only the simulation part of our entire system.
	Our system will furthermore offer the possibility to export these components as a combination of VHDL code and prepared netlists in order to create a bitfile to generate that sound live on an \ac{FPGA}. Additionally it is possible to change the behavior of some components at runtime by manipulating their input values through different sensors. 

\section{Reactable}

Another interactive music generator is Reactable, which is an innovative electronic musical instrument table where the user can deploy cubes on (also available as an smartphone app). Each cube represents either waveforms, loops, filters or other synthesizer components. These components can also be combined and linked. Similar to that, our editor can arrange different synthesizer components in order to generate music. Furthermore, we adapt the interactive input possibilities of the Reactable app. It is possible to connect nearly each component to an acceleration sensor or gyroscope of the phone the Reactable app is running on. This sensor measures the phone's tilt angle and thereby for instance can effect the music's pitch.