
\chapter{Workplan}
\label{chapter:Workplan}
%We are developing our system in an agile development process. Therefore we created different milestones, which are based on Use-Cases. Every milestone represents a functionally %working version of our whole system whereas the last milestone includes every single functionality. Testing is also done in between those milestones to guarantee a working system. 

\section{Overview}

%Mention our different milestones with their due dates. A Gantt Diagram would be fancy as hell.

The whole project is divided into multiple milestones. Each milestones consists of set of tasks. One milestone must be completed before antoher milestone can start.

\section{Milestones}

Five milestones are planned, where each milestones should be completed in approximately five to six weeks. Depending on the set of tasks in each milestone, some may consume more time, some less.

\subsection{Milestone 1 - Prototyping infrastructure/environment}

The fundamental infrastructure to achieve the project objectives is prototyped in the first milestone. This involves three basic tasks:

\begin{enumerate}
	\item prototype a graphical design environment (editor) for audio-generation as well as audio-processing components
	\item prototype a simulation environment to simulate a patch created with the editor
	\item prototype a hardware infrastructure such as digital-to-analog (dac) controller, (rotary) switches and an interface to a external sensors (Reconos)
\end{enumerate}

The goal of this milestone is not to have a complete working toolflow, but to have a quick start of the development. At the end of this milestone one should be able to open an editor and add/connect dummy components. The dummy components representing signal/audio processing components in later stages of development. The system of dummy components, refereed to as patch, can be exported to a simulator. The simulator will playback the patch on the editors host-pc. 
In addition to the software-side there will exist a working interface for the FPGA expansion board components at the end of this milestone, written in an HDL language. For testing purposes and later evaluation, it will be possible to interface the expansion board components with the schneckonos operating system. \todo{write reconos here}

In detail the basic tasks can be further subdivided:

\begin{enumerate}
%--------------------------------------------------------------------
	\item Editor prototype
		\begin{itemize}
			\item Implementation of a basic meta-model
			\item The editor prototype should implement at least the following basic features:
				\begin{itemize}
					\item Create/delete dummy sound-components
					\item Connect/disconnect dummy sound-components
					\item Save and load a model (patch)
					\item Serialize the patch to a human readable format (e.g XML)
				\end{itemize}
			\item Create at least two sound-components: a sound generator component (sine/square) and an audio output component
		\end{itemize}
%--------------------------------------------------------------------		
	\item Simulator prototype
		\begin{itemize}
			\item Evaluate JAVA-Sound API
			\item Design a framework
			\item Proof-of-concept implementation
			\begin{itemize}
				\item Import a patch-file
				\item Simulate system input
			\end{itemize}
		\end{itemize}
%--------------------------------------------------------------------
	\item Hardware prototype
		\begin{itemize}
			\item Setup Reconos development environment
			\item Prototype I/O expansion interface
			\begin{itemize}
				\item Create the DAC-controller and a correspondent testbench
				\item Rotary switch should work
			\end{itemize}
			\item Setup communication between a host-pc and reconos
			\item Encapsulate access to the XILINX FPGA-Toolchain (e.g. Eclipse plugin)
		\end{itemize}
\end{enumerate}


\subsection{Milestone 2 - Prototype of a digital synthesizer}

In the second milestone more features will be added to the whole system. The goal of this milestone should be, that a basic digital synthesizer system can be modeled by using the soundgate editor toolkit. The user will find basic components, known from digital signal processing (constants, adders, multiplicators, delays, ...), in the editors sound-component library as well as components which represent external interfaces (DAC-Audio-Output, basic sensors likes (rotary) switches). With these components it will be possible for a user to alter audio-signals in a signal processing manner and listen to the result on the COSMIC system. The interaction with the system is restricted to the basic sensors, mentioned in milestone one. The following list represents the set of tasks of achieve the the suggested goal.

	\begin{enumerate}
		\item Editor continued
			\begin{itemize}
				\item Add the ability to plugin new sound-components into the editor
				\item Add the ability to pass a patch to a codegeneration unit
				\item Add the ability to mark the input of a sound-component instance as an interactive parameter
			\end{itemize}
		\item Basic sensor interaction
			\begin{itemize}
				\item It should be possible adjust a parameter with a sensor, that is attached to the I/O expansion board
			\end{itemize}
		\item Codegeneration	
			\begin{itemize}
				\item Transform a patch to a synthesizable HDL description
				\item Import of existing HW-Framework components (e.g. DAC, switches)
			\end{itemize}
		\item Implement a set of basic atomic blocks
			\begin{itemize}
				\item wave generator (saw, square, sine)
				\item multiplication
				\item addition
				\item ramp generator
			\end{itemize}
	\end{enumerate}

\subsection{Milestone 3 - Polishing editing environment}

At the third milestone it should be possible for a musician to emulate the sound of an analog synthesizer. To achieve this, the editor will provide abstract sound-components a musician is familiar with (such as a mixer, waveform generators, filters, envelope generators,...), in the component library. With this a  musician will be able to create a patch and export it to the COSMIC system by simply start an export-process. The musician will be informed on the progress of the export-process. Furthermore a musician will be able to create his own sound-components in order to share them with others and import them accordingly.

	\begin{enumerate}
		\item Editor continued
			\begin{itemize}
				\item Add the ability to create composite sound-components
				\item Add the ability to import/export composite sound-components in a portable format
				%\item Add the ability to upload awesome sound-components to community platform %
			\end{itemize}
		\item Complex sensor interaction
			\begin{itemize}
				\item Create an Android application to stream sensor data from a smartphone to the COSMIC system.
			\end{itemize}
		\item Implement additional audio processing components	
			\begin{itemize}
				\item Filter: Low/High/Band-pass filters
				\item Mixer: An audio mixer with multiple inputs
				\item 
			\end{itemize}
		\item Implement a set of basic atomic blocks
			\begin{itemize}
				\item wave generator (saw, square, sine)
				\item multiplication
				\item addition
				\item ramp generator
			\end{itemize}
	\end{enumerate}
	
	
	%\begin{enumerate}
		%\item Implement additional audio processing blocks
		%\begin{itemize}
			%\item Arithmetic/Logic-Blocks (like "`Equals"')
			%\item Routing objects
			%\item midi processing blocks
			%\item Low/high-pass filters
			%\item delay
			%\item ...
		%\end{itemize}
	%\end{enumerate}
	
\subsection{Milestone 4 - Topic of milestone}

The forth milestone focus is to have an fully integrated system which reacts to human interaction. The toolflow is completely

\subsection{Milestone 5 - Documentation, Testing, Presentation and Drinking}

The fifth milestone is designed to polish different aspects of the system. One aspect targets the documentation of the editor, sound-components and the COSMIC system. Although all system-level components will be documented during development, these documents are just fragments of bigger picture. These fragments will be gathered and assembled into a well understandable form (e.g webpage or handbook). The target audience of these documentation will be musicians, and non-technical but interested end-users.  