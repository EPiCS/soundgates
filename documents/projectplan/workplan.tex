
\chapter{Workplan}

%We are developing our system in an agile development process. Therefore we created different milestones, which are based on Use-Cases. Every milestone represents a functionally %working version of our whole system whereas the last milestone includes every single functionality. Testing is also done in between those milestones to guarantee a working system. 

\section{Overview}

%Mention our different milestones with their due dates. A Gantt Diagram would be fancy as hell.

The whole project is divided into multiple milestones. Each milestones consists of set of tasks. One milestone must be completed before antoher milestone can start.

\section{Milestones}

Five milestones are planned, where each milestones should be completed in approximately five to six weeks. Depending on the set of tasks in each milestone, some may consume more time, some less.

\subsection{Milestone 1 - Prototyping infrastructure/environment}

The fundamental infrastructure to achieve the project objectives is prototyped in the first milestone. This involves three basic tasks:

\begin{enumerate}
	\item prototype a graphical design environment (editor) for audio-generation as well as audio-processing components
	\item prototype a simulation environment to simulate a patch created with the editor
	\item prototype a hardware infrastructure such as digital-to-analog (dac) controller, (rotary) switches and an interface to a external sensors (Reconos)
\end{enumerate}

The goal of this milestone is not to have a complete working toolflow, but to have a quick start of the development. At the end of this milestone one should be able to open an editor and add/connect dummy components. The dummy components representing signal/audio processing components in later stages of development. The system of dummy components, refereed to as patch, can be exported to a simulator. The simulator will playback the patch on the editors host-pc. 
In addition to the software-side there will exist a working interface for the FPGA expansion board components at the end of this milestone, written in an HDL language. For testing purposes and later evaluation, it will be possible to interface the expansion board components with the schneckonos operating system. \todo{write reconos here}

In detail the basic tasks can be further subdivided:

\begin{enumerate}
%--------------------------------------------------------------------
	\item Editor prototype
		\begin{itemize}
			\item Implementation of a basic meta-model
			\item Prototype editor:
				\begin{itemize}
					\item Create/delete atomic blocks
					\item Connect atomic blocks
					\item Save and load a model (patch)
					\item Serialize the patch to a human readable format (e.g XML)
				\end{itemize}
			\item Create at least two atomic blocks (a sound generator block and an audio output)
		\end{itemize}
%--------------------------------------------------------------------		
	\item Simulator prototype
		\begin{itemize}
			\item Evaluate JAVA-Sound API
			\item Design a framework
			\item Proof-of-concept implementation
			\begin{itemize}
				\item Import a patch-file
				\item Simulate system input
			\end{itemize}
		\end{itemize}
%--------------------------------------------------------------------
	\item Hardware prototype
		\begin{itemize}
			\item Setup Reconos environment
			\item Prototype I/O expansion interface
			\begin{itemize}
				\item DAC-Controller should work
				\item Rotary switch should work
			\end{itemize}
			\item Setup communication between a host-pc and reconos
			\item Encapsulate access to the XILINX FPGA-Toolchain (e.g. Eclipse plugin)
		\end{itemize}
\end{enumerate}


\subsection{Milestone 2 - Prototype of a digital synthesizer}

In the second milestone more features will be added to the whole system. The goal of this milestone should be, that a basic digital synthesizer system can be modeled by using the soundgate editor toolkit. The user will find basic components, known from digital signal processing (adders, multiplicators, delays, ...), in the editors soundcomponent library as well as components which represent external interfaces (DAC-Output, basic sensors likes (rotary) switches). The following list represents the set of tasks of achieve the the suggested goal

	\begin{enumerate}
		\item Editor continued
			\begin{itemize}
				\item Add the ability to create and import user-defined atomic blocks
				\item Add the ability to create composite blocks
			\end{itemize}
		\item Basic sensor interaction
			\begin{itemize}
			\item empty
			\end{itemize}
		\item Codegeneration	
			\begin{itemize}
				\item Transform a patch to a synthesizable HDL description
				\item Import of existing HW-Framework components (e.g. DAC, switches)
			\end{itemize}
		\item Implement a set of basic atomic blocks
			\begin{itemize}
				\item wave generator (saw, square, sine)
				\item multiplication
				\item addition
				\item ramp generator
			\end{itemize}
	\end{enumerate}

\subsection{Milestone 3 - Topic of milestone}
	
	\begin{enumerate}
		\item Implement additional audio processing blocks
		\begin{itemize}
			\item Arithmetic/Logic-Blocks (like "`Equals"')
			\item Routing objects
			\item midi processing blocks
			\item Low/high-pass filters
			\item delay
			\item ...
		\end{itemize}
	\end{enumerate}
	
%\subsection{Milestone 4 - Topic of milestone}