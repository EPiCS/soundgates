\chapter{Introduction}
\section{Introduction}
"`Soundgates – Interactive Music Synthesis on FPGAs"' is a project initiated by the Computer Engineering Group of the University of Paderborn, aiming to synthesise music on modern FPGAs. Furthermore, one or more users should use this interactive system, by means of manipulating the synthesised music with advanced sensors such as the acceleration sensor in modern smartphones. 

	
	
	
\section{Generative Music}



\subsection{Introduction}

As pointed out in \cite{Chandra2012}, there are many cultures where musicical experience is defined by people performing and people perceiving music. The only way to slightly exert influence on the performer's music is by cheering, shouting, etc. on a concert. The gap between performer and perceiver reaches its peak in the context of recorded music like CDs or MP3 files, where is no chance of manipulate the music. 
Actually, there is a rising trend for interacticing with music on your own or with a familiar social partner, like in the video game "`Guitar Hero"' \cite{Chandra2012, Planck2009}. Even without any musical knowledges or talent, it is more and more possible to interact, manipulate or even create music. Generative music combines the opportunities of making music without having knowledges of how to play an instrument and explicitly exert influence on music you hear. An early an popular example for generative music was Mozart's musical dice game. Given a set of small sections of music, it was randomly chosen which part was played next.



\subsection{Approaches} 

Genarative music (or algorithmic music) can be devided into the following subcategories \cite{Wooller2005}.

\subsubsection{Linguistic/Structural}

Recomposition of analyzed songs, using grammars or stochastic processes. 

\subsubsection{Creative/Procedural}

Randomly reorder pre-defined musical parts and play them.

\subsubsection{Biological Emergent}

(Simulation of) biological influences.

\subsubsection{Interactive/Behavioural}

No instruments - Recorded and filtered samples at the most. Synthesized music. Music generation fully controlled by user input/interaction. (see related work)

\section{Possible User Interactions}
Microsoft Kinect, acceleration sensor of a modern smartphone, etc.

\section{Structure}
Chapter 2... Chapter 3... and Chapter 4!