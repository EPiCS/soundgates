\chapter{Introduction}
	\label{chapter:Introduction}
	\section{Introduction}
	--- Von Hendrik, mit anderem Kram mergen ---
	
	
"`Soundgates – Interactive Music Synthesis on FPGAs"' is a project initiated by the Computer Engineering Group of the University of Paderborn, aiming to synthesise music on modern FPGAs. 
Furthermore, one or more users should use this interactive system, by means of manipulating the synthesised music with advanced sensors such as the acceleration sensor in modern smartphones.
	
	
	-----------------------------------------------
	
	
	\section{About this document}
	This document is the project plan of the project group ``Soundgates'' at the University of Paderborn. 
	It introduces the topic of our project group in Chapter \ref{chapter:Introduction} and the goals we want to achieve in Chapter \ref{chapter:Goals}.
	Chapter \ref{chapter:RelatedWork} covers systems similar to what we are going to create, escpecially the software called MAX.
	Chapters \ref{chapter:Organization} and \ref{chapter:Workplan} describe how the group will organize itself and the groups' milestones.

	Not only should this document serve as a basis for the later evaluation and grading by our professor, but also as a reference for our group during the main working phase.
	This working phase runs from ??.2013 to ??.2014
	\section{Definitions}
	 The following table displays some definitions, that will be used throughout this document.
	
	 \begin{tabular}[h]{|c|p{9.75cm}|}
	  \hline
	  Term & Definition \\
	  \hline
	  \hline
	  Composite Component & Definition \\\hline
	  Component & A basic building block to generate music ??Haben wir uns hier auf Block als Bezeichnung geeinight? Component eher im Sinne von Softwarekomponente\\\hline
	  Editor & The Editor is used to create a patch out of components to generate synthesizable code which can be put on a FPGA \\\hline
	  FPGA & Field Programmable Gate Array \\\hline
	  Patch & The entire system which consists of Components and Composite Components. A set of single Components can build a new Component \\\hline
	  Port & The interface from one Component to another one \\\hline
	  Simulation & The developed patch is played through the PC speakers \\\hline
	 \end{tabular}
	 
	\section{Project outline}
	  - Creating an editor for synthesizers. 
	  - Components of the synthesizer implemented in hardware (and software for simulation purpose)
	  - Implementation of generative music concepts
	  ------- Warum wollen wir das eigentlich tun? Was ist die Motivation dahinter (zumal es das aus Oslo ja aschon in Software gibt)
	
	\section{Introduction to sound synthsis}
	  - Artifical generation of sound
	  - Generation of basic waveforms
	    - More complex and rich patterns by methods like additive/subtractive/... synth 
	  - Further addition of filters etc
	  
	  - Originated from analog synthesizers, nowadays mostly digital. Software for general purpose PCs exist
	  
	  - Building patches: job of a sound designer, rather than a musician
	  
	\section{Generative music}
	
	
	  --- Stichpunkte Gunnar: ---
	  
	  - Creatioin of music depending on user interaction
	  - Users do not need to be musicians
	  - Playful approach to making music
	  - tightly connected to sound synthesis
	  

	\subsection{Introduction}

	As pointed out in \cite{Chandra2012}, there are many cultures where musicical experience is defined by people performing and people perceiving music. 
	The only way to slightly exert influence on the performer's music is by cheering, shouting, etc. on a concert. 
	The gap between performer and perceiver reaches its peak in the context of recorded music like CDs or MP3 files, where is no chance of manipulate the music. 
	Actually, there is a rising trend for interacticing with music on your own or with a familiar social partner, like in the video game "`Guitar Hero"' \cite{Chandra2012, Planck2009}. 
	Even without any musical knowledges or talent, it is more and more possible to interact, manipulate or even create music.
	Generative music combines the opportunities of making music without having knowledges of how to play an instrument and explicitly exert influence on music you hear.
	An early an popular example for generative music was Mozart's musical dice game. Given a set of small sections of music, it was randomly chosen which part was played next.



	\subsection{Approaches} 

	Genarative music (or algorithmic music) can be devided into the following subcategories \cite{Wooller2005}.

	\subsubsection{Linguistic/Structural}

	Recomposition of analyzed songs, using grammars or stochastic processes. 

	\subsubsection{Creative/Procedural}

	Randomly reorder pre-defined musical parts and play them.

	\subsubsection{Biological Emergent}

	(Simulation of) biological influences.

	\subsubsection{Interactive/Behavioural}

	No instruments - Recorded and filtered samples at the most. Synthesized music. Music generation fully controlled by user input/interaction. (see related work)

	\section{Possible User Interactions}
	Microsoft Kinect, acceleration sensor of a modern smartphone, etc.
	  
	\section{Employed systems}
	  \subsection{VHDL}
	    - Hardware components implemented in VHDL
	  \subsection{ReconOS}
		ReconOS is an Operating System which can be run on a softcore CPU on a FPGA. Through it's support for the Linux kernel it is possible to write applications which consist of Hard- and Software threads. Therefore the software threads can be used to communicate via sockets with sensors and hand over these values to the hardware threads, which are responsible for generating sounds.
	  \subsection{Eclipse/GMF}
		We avoid building a new graphical editor from ground up since this is an error prone process. Instead we rely on the Model Driven Software Development process for graphical editors which is provided by the GMF framework. Therefore we can generate an editor by specifing the Meta-Model. The graphical surface with our components are connections are the result. Additionally it is neccessary to provide functions for every component, so it will be possible to simulate the generated output.
	  
      