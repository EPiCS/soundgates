\chapter{Organization}
\label{chapter:Organization}
Our group and project follows an agile-like software development strategy where all members of the group have equal rights and
responsibilities when it comes to decisions that concern the project as a whole.
Beyond that we have one exception. One member of the group is representing the group externally, coordinates the team and observes deadlines.
This \emph{team coordinator} has a \emph{deputy} and has also the function of a tie-breaker in very rare cases when the group is not able to achieve a consensus.

Due to the different requirements of the final product we split our group in two teams.
The first team deals with software sub-projects like the editor or the simulation.
The second team focuses on the hardware that is needed.
The affiliation to one of these teams is based on previous experiences of the specific person.

The agile software development method uses iterations.
In this project, each iteration has a unique milestone and achieving this milestone is the goal in each iteration.

\section{Agile inspired development}
The following reasons inspired the agile-like development process that we discussed in the last section:

\begin{itemize}
\item Testing is an integrated part of the whole development process, which improves the
quality of the project.
\item Incremental development of the features by the iterative nature of agile software
development.
\item Regular release of a working project code at the end of each iteration.
\item Easy adaptation for any change in circumstances.\\
	  E.g. we do not know in advance which sensors we are going to use.
\item Self organizing development process provides more team member involvement and
friendly environment.
\item Less end to end project overhead is required.
\end{itemize}

\section{Meetings}
Since the beginning of our project group we have a mandatory team meeting every week.
Right now it is scheduled each Monday at six o'clock.

The mandatory meeting is attended by all members of the group.
In this meeting the current status of all tasks and the whole project is discussed.
Further tasks for the next week are developed by discussion among the team members.
The members also track the progress towards the approaching milestone on each of this mandatory meetings.
Tasks are created and the members are assigned to the tasks in this meeting. Tasks assignment is either done due to volunteering or by asking a member.
A member with knowledge about a given task by previous experience or a seminar topic is  more likely assigned to that task.

An additionally time slot on Wednesday eleven o'clock is reserved for each week.
In a normal week, the Wednesday time slot is open for all the team members to work together and experience team work, which helps to motivate each other.
In addition, the Wednesday time slot will also be used for the weekly team meeting if continuous cancellation of the mandatory team meeting happens due to holidays
or if more time for the mandatory meeting is needed.

The concrete time slots of the mandatory and the additional team meeting are valid until the new semester begins. The time of the meetings will then be discussed and maybe changed. Although the structure of one mandatory and one additional team meeting will remain during the whole project group.

Furthermore a single person or a subgroup consisting of two or maximal three members that is working on a certain task is given a time slot where he/they can work on the provided computers and \ac{FPGA}s.
The time slot must be kept by the person or the subgroup in order to ensure project progress and a controlled work flow.
Each member or subgroup has at least one time slot per week such that every group member is working each work on the project.

\section{Tools}

\section{Member role and specialization}

\section{Milestone presentation}