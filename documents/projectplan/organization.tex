\chapter{Organization}
\label{chapter:Organization}
Our group and project follows an agile-like software development strategy where all members of the group have equal rights and
responsibilities when it comes to decisions that concern the project as a whole.
Beyond that we have one exception. One member of the group is representing the group externally, coordinates the team and observes deadlines.
This \emph{team coordinator} has a \emph{deputy} and has also the function of a tie-breaker in very rare cases when the group is not able to achieve a consensus.

Due to the different requirements of the final product we split our group in two teams.
The first team deals with software sub-projects like the editor or the simulation.
The second team focuses on the hardware that is needed.
The affiliation to one of these teams is based on previous experiences of the specific person.

The agile software development method uses iterations.
In this project, each iteration has a unique milestone and achieving this milestone is the goal in each iteration.
A milestone represents a stage of development and is oriented at presentable prototypes. 

\section{Agile inspired development}
The following reasons inspired the agile-like development process that we discussed in the last section:

\begin{itemize}
\item Testing is an integrated part of the whole development process, which improves the
quality of the project.
\item Incremental development of the features by the iterative nature of agile software
development.
\item Regular release of a working project code at the end of each iteration.
\item Easy adaptation for any change in circumstances.\\
	  E.g. we do not know in advance which sensors we are going to use.
\item Self organizing development process provides more team member involvement and
friendly environment.
\item Less end to end project overhead is required.
\end{itemize}

\section{Meetings}
Since the beginning of our project group we have a mandatory team meeting every week.
Right now it is scheduled each Monday at six o'clock.

The mandatory meeting is attended by all members of the group.
In this meeting the current status of all tasks and the whole project is discussed.
Further tasks for the next week are developed by discussion among the team members.
The members also track the progress towards the approaching milestone on each of this mandatory meetings.
Tasks are created and the members are assigned to the tasks in this meeting. Tasks assignment is either done due to volunteering or by asking a member.
A member with knowledge about a given task by previous experience or a seminar topic is more likely assigned to that task.

An additional time slot on Wednesday eleven o'clock is reserved for each week.
In a normal week, the Wednesday time slot is open for all the team members to work together and experience team work, which helps to motivate each other.
In addition, the Wednesday time slot will also be used for the weekly team meeting if continuous cancellation of the mandatory team meeting happens due to holidays
or if more time for the mandatory meeting is needed.

The concrete time slots of the mandatory and the additional team meeting are valid until the new semester begins. The time of the meetings will then be discussed and maybe changed. Although the structure of one mandatory and one additional team meeting will remain during the whole project group.

Furthermore a single person or a subgroup consisting of two or maximal three members who are working on a certain task, chooses a time slot where he/they can work on the provided computers and \acp{FPGA}.
The time slots must be abided by the project members in order to ensure project progress and a controlled work flow.
Each member or subgroup has at least one time slot per week such that every group member is working on the project.

\section{Tools}

GitHub and its issue tracking functionalities are used to organize the project.
Source code, documents, presentations are shared using a GitHub repository.
The corresponding GitHub wiki is used to track statistics of the group meetings (attendees, time, \dots) and to track deadline misses of individual members.
This data is then used to see if one or more thresholds are exceeded such that the group can take actions against it and talk to the group member.
The GitHub wiki also contains the private contact details of all the team members.

We are plan to use GitHubs issue tracking functionality to represent our project plan with its milestones.
We also discussed Redmine but came to the conclusion that it is maybe more convenient to only have a single tool where everything is integrated. %only useful if it is actively used by all team members.
Since our main collaboration tool is GitHub we would also like to use this for issue- and bugtracking instead of another tools which might be to much overhead for us.
If GitHub proves to be not as useful as intended we will use Youtrack or Redmine. Youtrack has a GitHub Integration and will therefore be the second choice if GitHubs builtin functionalities will fail for our purposes. 
%Redmine is used to represent the whole project plan with all milestones.
%A Milestone is divided in task that are created and assigned to the team members according to the meeting outcomes.
%Each task has a deadline and the work progress of each member is tracked through Redmine.
%If a team member identifies a bug or a new task, then he creates the task or bug in Redmine and the details are immediately informed to all the other team members via email.
%Redmine is also used to track the overall project progress for each milestone and the tasks that are pending.

Skype is used by all the team members to have a real-time conversations medium between team members throughout the week.

A mailing list which includes all member email addresses is used for email communication between members.

A Google Calender, where all group members are registered, will be used to share all dates and events of the group.

\section{Member role and specialization}

Each team member in the project is responsible for both the project development and the project organization.
Each member's project development responsibilities include planning, analyzing, coding, documenting, testing, discussing and evaluating.
The project organization role of each member involves frequent interaction between all members, using
the project organization tools efficiently and update of current status wherever possible.
Each member updates his errorless compilable code to GitHub repository as frequent as possible, even if the code change is very little.
While committing the errorless code, the member also provides a brief information about the code change in the commit message.
The same commit rule is applied for any information that is committed to the repository.
Binaries are not allowed in the repository.
When a member has not attended a team meeting with or without excuse, the member
goes through the minutes of meeting in GitHub wiki mostly within the day of the meeting conducted.
The member himself has the whole responsibility to identify the task allocated in a team meeting.
He is also responsible to communicate with the other team members if any confusion arises about new tasks.
A task that was identified and assigned to a given member during a team meeting must also be created and assigned in the tracking tool by this member (if it has not been created so far).
Each member of a task updates its details in the tracking tool including problems faced, unique experiences and the solution identified.
If more than one solution is identified then the reason is provided for the chosen solution.
These information provided by the member are used by the team whenever a similar task has to be identified or when the corresponding part is explained in the documentation.
If a member identifies a bug while testing or coding, it is immediately reported in the tracking tool as a new bug, which will then be emailed to all the team members.

In conclusion all tasks in the tracking tool are frequently updated by the associated members.
Hence the status of all the tasks is known to the whole team in the team meeting even in the absence of members.
This helps to determine the project progress in the meeting and to identify problems.

\vspace{1cm}

\begin{center}
\begin{tabular}{c|c|c}
\hline
Member & Specialization & Status \\
\hline
\hline
Martin Boonk & Software, Simulation &  \\ 
\hline
Caius Cioran & Hardware, Sensors &  \\ 
\hline
Lukas Funke & Hardware & Deputy \\ 
\hline
Hendrik Hangmann & Hardware \\ %, \todo{?} &  \\ 
\hline
Andrey Pines & Software, Editor &  \\ 
\hline
Thorbj\"orn Posewsky & Hardware, Sensors & Team coordinator \\ 
\hline
Gunnar Wuellrich & Software \\ %, \todo{?} &  \\ 
\hline 
\end{tabular}
\end{center}

\section{Milestone presentation}

At the end of each iteration the current state is presented by the team members to the project mentors. This presentation has two parts.
In the first part of the milestone presentation, the plan for the completed milestone and its current status will be described in detail.
If any of the planned features are not completed within the milestone, the reasons for failure and current status are explained.
The members will also propose a revised plan for the incomplete feature for new milestone work plan (part two of the presentation).
Demos and screen shots of results from the milestone features will be provided wherever it is reasonable.
In the second part of the presentation, the revisited work plan for the next milestone is described.

The group members expect feedback from the project group mentors about the milestone and plans on the next milestone.
This represents the "continuous delivery" of an agile development process which ensures that the project does not deviate from its main goal,
that no features are forgotten or that the quality is obtained.