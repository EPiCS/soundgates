\chapter{Overall description}
	\section{Product perspective}
		\subsection{System interfaces}
		\subsection{User interfaces}
		\subsection{Hardware interfaces}
		\subsection{Software interfaces}
		\subsection{Operations}
	\section{Product functions}
	We provide a graphical editor for the synthesizer development. The library of the editor offers components like sound generators, filters etc. which the designer can choose from and connect to create an own synthesizer with specific sounds. Some components have attributes, which can be modified inside of the editor. It is possible to create composed components by combining several components into one. Additionally, the patch can be simulated in order to evaluate the created sound. Furthermore components can be labeled as interactable so the user can manipulate them at runtime. The developed patch can be exported as VHDL-code, which is synthesized and put on a FPGA.

The user of the synthesizer (e.g. a musician) interacts with the interactable components without knowing about the internal structure of the system. E.g. he can connect a MIDI device or specific sensors to the FPGA to modify input values and create different sounds. 
	\section{User characteristics}
	\section{Optional features}