\chapter{Specific requirements}

% TODO: definition in separate datei packen (und evtl. verbessern)
\newcommand{\funcRequirement}[6]{
\begin{center}
\begin{tabular}{|l|p{12cm}|}
\hline
Req. ID 		& #1 \\ \hline
Name 				& #3 \\ \hline
Description & #4 \\ \hline
Priority 		& #5 \\ \hline
Comment 		& #6 \\ \hline
\end{tabular}
\end{center}
}


	\section{Functional requirements}
	
%\funcRequirement{Requirement ID (0-...)}
%{Link to use case goes here}
%{Name}
%{Description goes here}
%{Priority goes here (low/mid/high}
%{Comment/Further description goes here}

	\funcRequirement{1}
	{Add component to a patch}
	{The soundgate designer (SD) user should be able to add components to a patch.}
	{high}
	{--}
	
  \funcRequirement{2}
	{Remove component from a path}
	{The SD user should be able to remove components from a patch.}
	{high}
	{--}
	
  \funcRequirement{3}
	{Modify attributes of a component}
	{The SD user should be able to modify static parameters of a component.}
	{mid}
	{The value of static parameters, like the channel of an audio output interface, are only allowed to be changed during design time.}
	
	\funcRequirement{4}
	{Add composite components}
	{The SD user should be able to create composite components.}
	{mid}
	{The priority is classified as medium, because a correct patch could be created without composite components. }
	
	\funcRequirement{5}
	{Add port to a composite component}
	{The SD user should be able to add a port to a composite component.}
	{mid}
	{Only composite components can have user defined ports. Ports of atomic components are fixed, such they cannot be added, removed or modified.}
	
	\funcRequirement{5.1}
	{Set port direction}
	{The SD user should be able to set the direction of a port.}
	{mid}
	{A port is either incoming or outgoing.}
	
	\funcRequirement{5.2}
	{Set port constraints}
	{The SD user should be able to add constraints to a port.}
	{mid}
	{This feature is subject to further discussions. In general a port has a certain datatype. The user could further reduce the datatype width by port constraints. As an example the frequenzy port of a waveform generator could have a user defined range of 10 .. 1000 Hz }
	
	\funcRequirement{6}
	{Connect Components by Links}
	{The SD user should be able to connect ports of multiple components.}
	{high}
	{The soundgate designer user can should be ablte to connect an outport of a component to in inport of another component. All other combinations are invalid.}
	
	\funcRequirement{7}
	{Validate Patch}
	{The SD user should be able to validate a before created patch.}
	{high}
	{The user will be notified on error or warnings (port datatype missmatch for e.g). There sould also be a life validation.}
	
	\funcRequirement{8}
	{Perform editor operation}
	{The SD user should be able to perform standard editor operation.}
	{high}
	{Standard editor operations are for instance cut, copy, paste, move, etc..}
	
	\funcRequirement{9}
	{Import/Export patch}
	{The SD user sould be able to export a before created patch.}
	{high}
	{The patch can be exported as single file. It can then be imported into another soundgate editor.}
	
	\funcRequirement{9}
	{Build patch}
	{The SD user should be able to build/synthesize a patch and download it afterwords to an FPGA}
	{high}
	{--}
	
	\funcRequirement{10}
	{Import/Export composite components}
	{The SD user should be able to export a composite component in order to provide this to other SD users. The SD user should also be able to import composite components which were created by other SD users.}
	{mid}
	{--}
	
	\funcRequirement{11}
	{Start/Stop/Pause simulation}
	{The SD user should be able to run a simulation of patch.}
	{high}
	{--}
	
	\funcRequirement{12}	
	{Record simulation}
	{The SD user and the customer should be able to record a running simulation.}
	{low}
	{The SD user as well as the customer should be able to record the output of a running simulation. The requirement is defined as low, because the simulation can be run at any time.}
	
	\funcRequirement{13}
	{Import patch}
	{The SD user should be able to import a patch into the simulation environment.}
	{high}
	{--}
	
	\section{Non-functional requirements}
	\section{External interfaces}
	\section{Design constraints}
	\section{Performance requirements}
	\section{Software system attributes}
		\subsection{Reliability}
		\subsection{Availability}
		\subsection{Security}
		\subsection{Maintainability}
		\subsection{Portability}
	\section{Other requirements}